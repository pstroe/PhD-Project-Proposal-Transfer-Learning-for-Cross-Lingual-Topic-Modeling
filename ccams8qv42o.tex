\section{Research Project}
\subsection{Introduction}
``Digitisation'' is a buzz-word in several research communities in present days. The Digital Humanities are no exception. With the amount of digitised content increasing every day, this abundance of data asks not only for innovative, but also efficient ways of analyses and tools for exploration. Once these tools are in place, digital humanists can use them for their manifold research interests and hence knowledge discovery. As such, digitisation has the potential to serve as a catalyst in research and supports researchers in their undertakings.

I say ``potential'' since digital content alone does not yet facilitate research. In the meantime, digitisation has produced huge data silos. At best, the digitised content is available over a web interface and can thus be browsed quite conveniently. This has the positive effect that scientists nowadays can access their data online. However, dedicated systems to analyse and explore the data are missing. The research questions of scientists are often more complex and hence basic search functions do not suffice.

This calls for elaborate information retrieval (IR) systems which assist scientists in conducting their research. Of course, the functionalities of such systems depend on the data to be analysed. Currently, the most often digitised data are textual data. Libraries and archives play a key role here. They have collected huge amounts of historical documents over the past centuries. The majority of these collections are stored in shelves, doomed to fall into oblivion. Nevertheless, or rather therefore, more and more of these collections are nowadays digitised. In a way, they act as gatekeepers and their digitisation strategy influences which information is made publicly available or otherwise accessible online.  

Among written texts, it is mainly books which are digitised. However, there is another text category which currently undergoes massive digitisation processes: newspapers \citep{lansdall-welfarecontent2017,prestondigital2016,binghamdigitization2010}. There are several reasons why we see more and more libraries and archives digitising their material. One is the preservation of delicate documents, yet another is to make certain data more popular, or more easily accessible for different user groups. This proves to be a valuable opportunity for scholars in the digital humanities, like e.~g.~historians, sociologists, and also linguists. Newspaper texts as a scientific resource are of equal importance to each of the aforementioned group of scholars. Newspapers can be seen as a mirror of society. They report daily not only about local, but also about global events and are part of the cultural, social, and political heritage. Most newspaper archives cover several centuries. As such, they serve as an important primary source for historians who want to gain new insights on our past, for sociologists who wish to shed light on social movements, or for linguists who seek to examine how language has changed over time.

In order to allow researchers to work with the digitised data, it must be carefully processed. There exist several information retrieval techniques to analyse the data. Newspapers are particularly interesting texts to analyse since they are already structured by nature. Most newspapers follow a categorisation scheme which assigns each article to a category like ``Economics'' or ``International Affaires''. This convention was not common in the 19th century, though, where newspapers rather tended to publish news as they received them. The challenge we thus face today is to index data for researchers in the humanities so they can easily find content they are interested in. 

One technique that has gained popularity in the digital humanities over the recent years and is used in order to discover the underlying structure of textual data is topic modeling. Topic modeling is an unsupervised generative modeling method which finds topic distributions in documents. These topic distributions are characterised by word distributions, hence each topic consists of a collection of words which should describe a topic. As such, and this is an often stated need by historians, having identified a relevant article, topic models allows the researcher to find similar pages or articles based on the search result.

Applying topic models in the digitial humanities has some shortcomings: 

\begin{enumerate}
    \item Errors from Optical Character Recognition (OCR) directly influence the results. There are either topics which are dominated by them, or they distort the word distributions in a way that they pretend to characterise a topic.
    \item Topic modeling algorithms require a parameter for the number of topics to be inferred from the text. The number of topics can vary from one newspaper collection to another. This would require us to train a topic model for each collections, which is a time consuming undertaking. Moreover, the choice about the number of topics determines how fine or coarse grained a search in a collection can be. The number of topics one want to look at thus directly depends on the researcher working with the texts.
    \item When working with newspaper collections in different languages, we would need to train a topic model for each language, which again is very costly.
    \item Although not restricted to the digital humanities, there are no direct measures to 
\end{enumerate}

Another
Since Switzerland is a multilingual country with four official languages (German, French, Italian, Rumansh), the issue of linking topics cross-linguistically quickly arises. After all, it could be possible that the media in the different language regions of Switzerland treated the women's suffrage differently, which is why the modeling of topic relations across languages is of particular interest. Additionally, such a linking over languages and time would also bring the attitudes of different newspapers from different regions to light.

Hence, this research project will not only be occupied with topic modeling \textit{per se}, but also with finding cross-lingual topic relations, while at the same time putting them into a historical perspective.

\subsection{Problem statement}
This research project will mainly deal with Swiss newspaper texts. The nature of the data treated in this research project puts additional complexities to the task of topic modeling. On one hand, the newspaper data is diachronic and covers a time period of roughly 200 years. Hence, not only the problem of different orthographies arises, but also the issue of meaning changes, which directly influence topic modeling techniques. On the other hand, the multilinguality (German, French, a small portion of Italian) of the texts adds an additional twist. To summarise, this research project faces the following challenges and sub-challenges:

\begin{itemize}
	\item diachronicity
	\begin{itemize}
		\item meaning changes of words
		\item orthographic changes
		\item normalisation (specification needed)
	\end{itemize}
	\item multilinguality
	
\end{itemize}

\paragraph{Research question}
How can cross-lingual topic relations be modeled across time?

\paragraph{Significance}

\section{Background and State-of-the-Art}

\subsection{Topic modeling}
Topic modeling is a well researched field. The most often used techniques are Latent Semantic Analysis (LSA) \citep{deerwesterscottindexing1990}, Probabilistic Latent Semantic Indexing \citep{hofmannprobabilistic1999}, and Latent Dirichlet Allocation (LDA) \citep{bleidynamic2006}. In principle, LSA and LDA represent probabilistic methods to identify possible topics in a text. They take into account the distribution of words in a text and try to narrow down the topic to words which represent the contents of a text best.

Since the words which describe a topic are expected to be semantically related, techniques like LDA profit from the advent of word embeddings \citep{mikolovdistributed2013,bengioneural2003}, which capture the semantic similarity of words in a low-dimensional vector space. \citet{dasgaussian2015} successfully incorporate word embeddings into Gaussian LDA and report that their technique outperforms standard LDA especially when dealing with out-of-vocabulary words.

With the integration of word embeddings into standard LSA and LDA methods, (deep) neural network techniques find their way into the field of topic modeling. \citet{ganscalable2015} introduce Deep Poisson Factor Analysis, which employs sigmoid belief networks and restricted Boltzmann machines in order to assign topics to text. They find that their method is superior to topic modeling using LDA. \citet{cao_novel_2015} propose the feedforward Neural Topic Model, where the different layers correspond to the different distributions in regular topic modeling. Yet another approach take \citet{miao_neural_2016}. They model the topics using a variational auto-encoder, thereby looking at the problem of topic modeling as text generation (which it actually is).

\subsection{Multilingual data}
In the course of this research project, we will also work with parallel data, that is newspapers, periodicals, and other publications which are translated from one language into at least one other language. Once we apply topic modeling to multilingual data, we are able to cluster events, classify documents, become engaged in cross-lingual semantic similarity of words, as well as in cross-lingual information retrieval, as \citet{vulicprobabilistic2015} point out. They further remark that the majority of multilingual topic modeling techniques relies on standard LDA, LSA, and pLSA, all of which appear to be fairly common and widely used for topic modeling anyway \citep{dumaisautomatic1997,mimnopolylingual2009,zhangcrosslingual2010,nicross2011}. Prerequisites for multilingual topic modeling are good sentence and word alignments, as well as an appropriate set of latent cross-lingual topics.

Since the data with which we will experiment stems from various sources, one big challenge will be to find one (or maybe multiple) sets of latent cross-lingual topics.  ...


\section{Research Project Plan}
Table \ref{researchplan} presents a detailed version of the research project plan.
\begin{table}
\centering
\resizebox{\textwidth}{!}{%
\begin{tabular}{l|llcllllcllllcll|}
\cline{2-16}
 & \multicolumn{3}{c|}{\textbf{2017}} & \multicolumn{5}{c|}{\textbf{2018}} & \multicolumn{5}{c|}{\textbf{2019}} & \multicolumn{2}{c|}{\textbf{2020}} \\ \cline{2-16} 
 & \multicolumn{1}{l|}{\textbf{Sep-Nov}} & \multicolumn{2}{l|}{\textbf{Dec-Feb}} & \multicolumn{1}{l|}{\textbf{Mar-May}} & \multicolumn{1}{l|}{\textbf{Jun-Aug}} & \multicolumn{1}{l|}{\textbf{Sep-Nov}} & \multicolumn{2}{l|}{\textbf{Dec-Feb}} & \multicolumn{1}{l|}{\textbf{Mar-May}} & \multicolumn{1}{l|}{\textbf{Jun-Aug}} & \multicolumn{1}{l|}{\textbf{Sep-Nov}} & \multicolumn{2}{l|}{\textbf{Dec-Feb}} & \multicolumn{1}{l|}{\textbf{Mar-May}} & \textbf{Jun-Aug} \\ \cline{2-16} 
\textbf{Preparation} &  & \multicolumn{2}{l}{} &  &  &  & \multicolumn{2}{l}{} &  &  &  & \multicolumn{2}{l}{} &  &  \\
Literature review & \cellcolor[HTML]{3166FF} & \multicolumn{2}{l}{\cellcolor[HTML]{3166FF}} & \cellcolor[HTML]{ECF4FF} & \cellcolor[HTML]{ECF4FF} & \cellcolor[HTML]{ECF4FF} & \multicolumn{2}{l}{\cellcolor[HTML]{ECF4FF}} & \cellcolor[HTML]{ECF4FF} & \cellcolor[HTML]{ECF4FF} & \cellcolor[HTML]{ECF4FF} & \multicolumn{2}{l}{\cellcolor[HTML]{ECF4FF}} & \cellcolor[HTML]{ECF4FF} &  \\
Investigate/Implement current methods & \cellcolor[HTML]{3166FF} & \multicolumn{2}{l}{\cellcolor[HTML]{3166FF}} &  &  &  & \multicolumn{2}{l}{} &  &  &  & \multicolumn{2}{l}{} &  &  \\
\textbf{Research Tasks} &  & \multicolumn{2}{l}{} &  &  &  & \multicolumn{2}{l}{} &  &  &  & \multicolumn{2}{l}{} &  &  \\
Topic modeling with word embeddings &  & \multicolumn{2}{l}{} & \cellcolor[HTML]{34FF34} & \cellcolor[HTML]{34FF34} &  & \multicolumn{2}{l}{} &  &  &  & \multicolumn{2}{l}{} &  &  \\
Topic modeling with deep learning &  & \multicolumn{2}{l}{} &  &  & \cellcolor[HTML]{34FF34} & \multicolumn{2}{l}{\cellcolor[HTML]{34FF34}} &  &  &  & \multicolumn{2}{l}{} &  &  \\
Relational topic modeling &  & \multicolumn{2}{l}{} &  &  &  & \multicolumn{2}{l}{} & \cellcolor[HTML]{34FF34} & \cellcolor[HTML]{34FF34} &  & \multicolumn{2}{l}{} &  &  \\
Cross-linguistic topic modeling &  & \multicolumn{2}{l}{} &  &  &  & \multicolumn{2}{l}{} &  &  & \cellcolor[HTML]{34FF34} & \multicolumn{2}{l}{\cellcolor[HTML]{34FF34}} & \cellcolor[HTML]{34FF34} &  \\
Historical/Dynamic Topic Modeling & & & & & \cellcolor[HTML]{34FF34} &\cellcolor[HTML]{34FF34} &\cellcolor[HTML]{34FF34} &\cellcolor[HTML]{34FF34} &\cellcolor[HTML]{34FF34} &\cellcolor[HTML]{34FF34} &\cellcolor[HTML]{34FF34} &\cellcolor[HTML]{34FF34} &\cellcolor[HTML]{34FF34} & \\
\textbf{Milestones} &  & \multicolumn{2}{l}{} &  &  &  & \multicolumn{2}{l}{} &  &  &  & \multicolumn{2}{l}{} &  &  \\
Paper drafts &  & \multicolumn{2}{c}{} & \multicolumn{1}{c}{\textbf{}} & \multicolumn{1}{c}{\textbf{X}} & \multicolumn{1}{c}{\textbf{}} & \multicolumn{2}{c}{\textbf{X}} & \multicolumn{1}{c}{\textbf{}} & \multicolumn{1}{c}{\textbf{X}} & \multicolumn{1}{c}{\textbf{}} & \multicolumn{2}{c}{\textbf{X}} & \multicolumn{1}{c}{\textbf{}} & \multicolumn{1}{c|}{} \\
Annual reports &  & \multicolumn{2}{l}{} &  &  & \textbf{X} & \multicolumn{2}{l}{} &  &  & \textbf{X} & \multicolumn{2}{l}{} &  &  \\
Dissertation compilation &  & \multicolumn{2}{l}{} &  &  &  & \multicolumn{2}{l}{} &  &  &  & \multicolumn{2}{l}{} & \cellcolor[HTML]{FE0000} & \cellcolor[HTML]{FE0000} \\ \cline{2-16} 
\end{tabular}%
}
\caption{Research Project Plan: An overview of the research program proposed.}
\label{researchplan}
\end{table}

\section{Research Project Details}

\paragraph{Research Project Communication and Deliverables:}
The majority of the outputs of this research plan will be published in journals and at conferences (see the research project plan in Table \ref{researchplan}). The aim is to publish at A-level conferences or in A-level journals. The following deliverables can be defined (non-exhaustive list):

\begin{itemize}
	\item Topic model of newspaper articles which can be integrated in the search engine which will be realised within the Sinergia project
	\item Software to apply (historical) topic modeling on multilingual text collections.
	\item Software relying on relational topic modeling to build a document network.
\end{itemize}

\paragraph{Data Management:}
As detailed in the Sinergia proposal, the EPFL will be responsible for the management of the data. Where licensing permits, the candidate will make new data sets and tools available through his website and/or Github. In addition, we will also release the data set, both in normalised an enriched versions.

\paragraph*{Skills Audit:}
A skills audit for the skills required in this project is shown in Table \ref{skillsaudit}.

\begin{table}
\centering
\resizebox{\textwidth}{!}{%
\begin{tabular}{|c|c|c|c|c|c|c|}
\hline
\multicolumn{1}{|l|}{\multirow{2}{*}{Professional and Research Skills}} & \multicolumn{4}{c|}{Rating} & \multicolumn{1}{l|}{\multirow{2}{*}{Evidence}} & \multicolumn{1}{l|}{\multirow{2}{*}{Plan for acquisition}} \\ \cline{2-5}
\multicolumn{1}{|l|}{} & none & basic & competent & proficient & \multicolumn{1}{l|}{} & \multicolumn{1}{l|}{} \\ \hline
\begin{tabular}[c]{@{}c@{}}Understanding of mathematics\\ required for this area\\ (Probability, Linear Algebra,\\ Calculus)\end{tabular} &  & X &  &  & \begin{tabular}[c]{@{}c@{}}Completion of introductory\\ calculus course\end{tabular} & \begin{tabular}[c]{@{}c@{}}Aim must be to move\\ to competent. Taking\\ self-study courses\\ on Coursera or edX.\end{tabular} \\ \hline
\begin{tabular}[c]{@{}c@{}}Using of programming languages\\ for this project (Python,\\ probably Scala)\end{tabular} &  &  & X &  & \begin{tabular}[c]{@{}c@{}}Completion of programming\\ courses related to\\ Computational Linguistics,\\ plus additional self-tuition\end{tabular} & No action required \\ \hline
\begin{tabular}[c]{@{}c@{}}Knowledge and\\ understanding of\\ topic modeling\end{tabular} &  & X &  &  & Introductory MA course & \begin{tabular}[c]{@{}c@{}}Read up on recent\\ literature, aim must\\ be proficient (by the\\ end of this research\\ project)\end{tabular} \\ \hline
\begin{tabular}[c]{@{}c@{}}Knowledge and\\ understanding of\\ machine learning\end{tabular} &  & X &  &  & Introductory MA course & \begin{tabular}[c]{@{}c@{}}Take a machine\\ learning course (UZH,\\ ETH, or self-tuition).\end{tabular} \\ \hline
\begin{tabular}[c]{@{}c@{}}Knowledge and\\ understanding of\\ deep learning\end{tabular} &  & X &  &  & Participation in Reading Group & \begin{tabular}[c]{@{}c@{}}Take a deep\\ learning course (ETH)\end{tabular} \\ \hline
\begin{tabular}[c]{@{}c@{}}Understanding and \\ application of data\\ collection and\\ analysis methods\end{tabular} &  &  & X &  & \begin{tabular}[c]{@{}c@{}}Participation in Text+Berg\\ and Bulletin4Corpus projects\end{tabular} & No action required. \\ \hline
\end{tabular}%
}
\caption{Skills audit}
\label{skillsaudit}
\end{table}

\section{Research Training}
The research training plan in the following subsection serves as an overview on planned activities in order to obtain the required ECTS (12 in total). The plan is to fulfil this requirement as early as possible, such that in the following, the focus is on the research plan itself.

\subsection{Research Training Plan}
The research training plan with the planned activities I plan for my PhD research follows in Table \ref{researchtraining}.

\begin{table}
\label{researchtraining}
\centering
\begin{tabular}{|l|l|c|c|}
\hline
Date       & Activity                                                                                                   & \multicolumn{1}{l|}{Training} & \multicolumn{1}{l|}{Milestone} \\ \hline
15/08/2017 & Research Proposal                                                                                          & \multicolumn{1}{l|}{}         & X                              \\ \hline
01/09/2017 & Start of the project                                                                                       &                               &                                \\ \hline
19/09/2017 & Kolloquium (3 ECTS)                                                                                        & X                             & \multicolumn{1}{l|}{}          \\ \hline
25/09/2017 & Deep Learning course ETH (4 ECTS)                                                                          & X                             &                                \\ \hline
05/10/2017 & \begin{tabular}[c]{@{}l@{}}Publish or perish: Designing\\ research for publication (1 ECTS)\end{tabular}   & X                             &                                \\ \hline
06/11/2017 & \begin{tabular}[c]{@{}l@{}}Scientific publication: discover, manage\\ \& disseminate (1 ECTS)\end{tabular} & X                             & \multicolumn{1}{l|}{}          \\ \hline
28/11/2017 & \begin{tabular}[c]{@{}l@{}}Resource-focused stress\\ management (1 ECTS)\end{tabular}                      & X                             &                                \\ \hline
SS18       & GRC course 1 (1 ECTS)                                                                                      & X                             &                                \\ \hline
SS18       & GRC course 2 (1 ECTS)                                                                                      & X                             &                                \\ \hline
01/03/2018 & \begin{tabular}[c]{@{}l@{}}Confirmation of candidature\\ (Doktoratsvereinbarung)\end{tabular}              &                               & X                              \\ \hline
01/09/2018 & Annual report year 1                                                                                       & \multicolumn{1}{l|}{}         & X                              \\ \hline
01/09/2019 & Annual report year 2                                                                                       & \multicolumn{1}{l|}{}         & X                              \\ \hline
01/03/2020 & Dissertation draft                                                                                         & \multicolumn{1}{l|}{}         & X                              \\ \hline
01/09/2020 & Dissertation submitted for examination                                                                     & \multicolumn{1}{l|}{}         & X                              \\ \hline
\end{tabular}
\caption{Research training plan: planned activities}
\end{table}

\subsection{Working Hours}
The candidate dedicates 100\% of his working hours (25.2 hours per week) within this proposed research program. The supervision of BA or MA projects, as well as the tutoring of lectures or seminars are possible, but restricted to one supervision/teaching assistant post per semester.

\section{Supervision}

\paragraph{Principal \& Coordinating Supervisor: Prof.~Dr.~Martin Volk}

\begin{itemize}
	\item Directing overall research program
	\item Review research outputs
	\item Provide regular feedback, on both overall, and current subproject progress
\end{itemize}

\paragraph{Co-Supervisor: Dr.~Simon Clematide}

\begin{itemize}
	\item Supervise overall research program
	\item Provide expertise in text mining (especiall topic modeling) and deep learning techniques
	\item Provide regular feedback, on both overall, and current subproject process
\end{itemize}